% RMIT University School of CS&IT
% Minor thesis template
% S.M.M. (Saied) Tahaghoghi, 2004
\documentclass[11pt,twoside]{report}
\usepackage{a4wide,caption,epsfig,fancyheadings,natbib,url}

% Place the correct values here
%Set to the original submission date when submitted amended thesis
\newcommand{\SubmissionDate}{\today}
\newcommand{\student}{Qi Xiong}
\newcommand{\supervisor}{Hai Dong}
\newcommand{\topic}{Enhance the performance of Github repository recommendation by using a Knowledge Graph Neural Network based approach}
\newcommand{\school}{School of Computer Science and Information Technology}
\newcommand{\program}{Masters of Applied Science (Information Technology)}
\newcommand{\institution}{Royal Melbourne Institute of Technology}

% Use the remark command to highlight text for discussion
\newcommand{\remark}[1]{{\bf \em [\marginpar{$\Leftarrow$}#1]}}

\renewcommand{\leftmark}{\student}
\renewcommand{\rightmark}{\topic}
\renewcommand{\headrulewidth}{0pt}
\setlength{\parindent}{0pt}
\setlength{\parskip}{1.5ex plus 0.3ex}

% This is the line spacing - set to 2 for draft submission to
% supervisor, 1.3 for the final submission
\renewcommand{\baselinestretch}{1.00}

\renewcommand{\captionfont}{\it}
\raggedbottom

%For Natbib Author, year citation format
% - the opening bracket symbol, default = (
% - the closing bracket symbol, default = )
% - the punctuation between multiple citations, default = ;
% - the letter n for numerical style, or s for numerical superscript
%   style, any other letter for author year, default = author year;
% - the punctuation that comes between the author names and the year
% - the punctuation that comes between years or numbers when common author lists are suppressed (default = ,);
\bibpunct{[}{]}{;}{a}{,}{;}


\begin{document}

\title{{\Large\bf \topic}}
\author{
A minor thesis submitted in partial fulfilment of the requirements for the degree of
\\\program\\*[10mm]
%\epsfig{figure=Figs/rmit-coa.epsf,width=5cm}
\\\student
\\\school
\\Science, Engineering, and Technology Portfolio,
\\\institution
\\Melbourne, Victoria, Australia
}
\maketitle
\thispagestyle{empty}

\chapter*{Declaration}

This thesis contains work that has not been submitted previously, in
whole or in part, for any other academic award and is solely my
original research, except where acknowledged.

This work has been carried out since July 2021, under the
supervision of {\supervisor}.

\paragraph{}
\vspace{5cm}\noindent \\\student \\
\school\\
\institution\\
\SubmissionDate

\pagenumbering{roman}

\chapter*{Acknowledgements}

I sincerely thank my supervisor Hai Dong for his patience and support. Without his support, this work would be difficult to continue. I also thank the GH Archive project for oppening their records of the public GitHub activities from developers and repositories to everyone. Without this open project, this work would also be hard to continue. Finally, I thank my families and friends for their help and supports.

\tableofcontents
\listoffigures
\listoftables

\pagenumbering{arabic}

\chapter{Introduction}
GitHub \footnote{https://github.com} is an open-source community that enables developers to manage their repositories via a version control system named Git \footnote{https://git-scm.com/}. Developers can also share their project and contribute to other developers’ project in this community.

The recommender system for GitHub repositories is important. Developers tend to find repositories which are similar to the one they are working on to find functions or ideas which can be reused. Although the search engine provided by GitHub can help find similar repositories, it is always not accurate because it only uses the title of repositories rather than the whole details according to \cite{xu_repersp_2017}. While recommender systems can recommend the most relevant repositories to users based on the preference of users and repositories.

Most recommender systems have data sparsity and cold start issues. Due to the long tail effect, few popular products have the most ratings while lots of others have fewer ratings. As a result, the constructed user-item matrix is sparse. It is inefficient for a model to deal with a sparse matrix. If a user does not rate any items then it is hard for the recommender system to recommend items to that user. This is the cold start issue. Because in the beginning, there are always few ratings. Researchers have proposed various approaches to tackle these issues.

One of the approaches is to take the side information which was investigated by \cite{jonschkowski_patterns_2016} of users and items into consideration. Side information can be efficiently integrated into a knowledge graph neural network. With more information can be used to determine an user's preference, a recommender system thus can fight for data sparcity issues and improve the performance. The improvements are promising according to \cite{zhang_knowledge_2020} which applied side information and knowledge graph neural network in mobile APP recommendations.

Many existing works have proposed various GitHub recommender systems. But non of them can deal with the cold start problem because they ignored the use of the side information for the recommender system.

This work proposed a knowledge graph neural network based model to recommend Github repositories to developers. The performance and the ability to fight data sparcity issues are promising compared with traditional recommender systems.

\chapter{Literature review}

\section{Recommender systems}

\section{Graph neural networks}

\section{Another Section}

\appendix
\chapter{Testbed Configuration}

\bibliographystyle{abbrvnat}
\bibliography{Bib/strings,Bib/main}
\end{document}
